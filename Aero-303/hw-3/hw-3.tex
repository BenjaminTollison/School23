\documentclass[12pt]{exam}
\usepackage[utf8]{inputenc}

\usepackage{amsmath,amstext,amsthm,amssymb,amsxtra, graphicx}
\usepackage[top=1.5in, bottom=1.5in, left=1.25in, right=1.25in]	{geometry}
%\usepackage[normalem]{ulem}
\usepackage{txfonts} % pxfonts txfonts 
\usepackage[T1]{fontenc}
\usepackage{lmodern}
\renewcommand*\familydefault{\sfdefault}
\usepackage{euler}   % better than the option below
\usepackage{pdfsync}
\usepackage{multicol}
\newcommand{\ci}[1]{_{ {}_{\scriptstyle #1}}}
\graphicspath{ {images/} }


\newcommand{\norm}[1]{\ensuremath{\left\|#1\right\|}}
\newcommand{\abs}[1]{\ensuremath{\left\vert#1\right\vert}}
\newcommand{\ip}[2]{\ensuremath{\left\langle#1,#2\right\rangle}}
\newcommand{\p}{\ensuremath{\partial}}
\newcommand{\pr}{\mathcal{P}}

\newcommand{\pbar}{\ensuremath{\bar{\partial}}}
\newcommand{\db}{\overline\partial}
\newcommand{\D}{\mathbb{D}}
\newcommand{\B}{\mathbb{B}}
\newcommand{\Sp}{\mathbb{S}}
\newcommand{\T}{\mathbb{T}}
\newcommand{\R}{\mathbb{R}}
\newcommand{\Z}{\mathbb{Z}}
\newcommand{\C}{\mathbb{C}}
\newcommand{\N}{\mathbb{N}}
\newcommand{\Q}{\mathbb{Q}}
\newcommand{\mQ}{\mathcal{Q}}
\newcommand{\mS}{\mathcal{S}}
\newcommand{\scrH}{\mathcal{H}}
\newcommand{\scrL}{\mathcal{L}}
\newcommand{\td}{\widetilde\Delta}
\newcommand{\pw}{\text{PW}}
\newcommand{\esup}{\text{ess.sup}}
\newcommand{\Tn}{\mathcal{T}_n}
\newcommand{\Bn}{\mathbb{B}_n}
\newcommand{\rt}{\mathcal{O}}
\newcommand{\avg}[1]{\langle #1 \rangle}
\newcommand{\one}{\mathbbm{1}}
\newcommand{\eps}{\varepsilon}
\newcommand{\grad}{\nabla}

\newcommand{\La}{\langle }
\newcommand{\Ra}{\rangle }
\newcommand{\rk}{\operatorname{rk}}
\newcommand{\card}{\operatorname{card}}
\newcommand{\ran}{\operatorname{Ran}}
\newcommand{\osc}{\operatorname{OSC}}
\newcommand{\im}{\operatorname{Im}}
\newcommand{\re}{\operatorname{Re}}
\newcommand{\tr}{\operatorname{tr}}
\newcommand{\vf}{\varphi}
\newcommand{\f}[2]{\ensuremath{\frac{#1}{#2}}}

\newcommand{\hn}{{3}}
\newcommand{\dd}{{07\-29}}
\newcommand{\class}{Aero 303}
\newcommand{\term}{Summer 2023}
\newcommand{\examnum}{Homework \hn}
\newcommand{\examdate}{}
\newcommand{\timelimit}{75 Minutes}
\newcommand{\vc}[3]{\langle #1,#2,#3\rangle}
\newcommand*{\vv}[1]{\vec{\mkern0mu#1}}
\newcommand{\bv}[1]{\boldsymbol{#1}}
\newcommand{\hide}[1]{}
\newcommand{\uvec}[1]{\boldsymbol{\hat{\textbf{#1}}}}
\newcommand{\vex}[1]{\boldsymbol{{\textbf{#1}}}}
\newcommand{\px}{\frac{\partial}{\partial x}}
\newcommand{\py}{\frac{\partial}{\partial y}}
\newcommand{\pt}{\frac{\partial}{\partial t}}
\newcommand{\pxx}{\frac{\partial^2}{\partial x^2}}
\newcommand{\pyy}{\frac{\partial^2}{\partial y^2}}
\newcommand{\ptt}{\frac{\partial^2}{\partial t^2}}


\pagestyle{head}
\firstpageheader{}{}{}
\runningheader{\class}{ Page \thepage\ of \numpages}{\examnum}
\runningheadrule

\makeatletter
\renewcommand*\env@matrix[1][*\c@MaxMatrixCols c]{%
  \hskip -\arraycolsep
  \let\@ifnextchar\new@ifnextchar
  \array{#1}}
\makeatother

\printanswers
\begin{document}

\noindent
\begin{tabular*}{\textwidth}{l @{\extracolsep{\fill}} r @{\extracolsep{6pt}} l}
\textbf{\class} & \textbf{Name:} & \makebox[2in]{\bf{Benjamin Tollison}}\\
\end{tabular*}\\
\rule[2ex]{\textwidth}{2pt}
%
\begin{questions}
\begin{question}
In contrast to static flow quantities, total/stagnation/sonic flow quantities are the aerodynamicist's best friend because the latter are conserved under certain conditions.\\
\indent a) What is a static quantity\\
\indent b) What operation do we apply to a fluid particle in order to let the associated flow properties attain their total values\\
The operation described for b) can be performed in many ways. Which property do we fix while
applying the operation if the property of interest is the:\\
\indent c) total pressure, \(p_0\), or the total density, \(\rho_0\)\\
\indent d) total temperature, \(T_0\)\\
\indent e)  Consider that we want to calculate the temperature at the stagnation point of an airfoil at
supersonic flow conditions, \(M_\infty\) > 1, i.e. the flow field will feature a shock wave ahead of the
airfoil. Can we make use of the expression for this application?:\[\frac{T_0}{T_\infty} = 1 + \frac{\gamma}{2}M_\infty^2\]
\end{question}
\begin{solutionorbox}[\stretch{1}]
\\a) Static quantities are the scalars of at a particular point in space independent of the reference frame, and do not take into account the flow properties of the fluid.
\\b) We find the kinetic energy or momentum of the point in relation to a reference frame
\\c) We assume no entropy is produced as we slow the flow down to have zero speed, and we also use the ideal gas law.
\\d) We don't have to assume the flow is isentropic, but that no heat is transferred to the steady flow
\\e) Yes, because entropy is produced as we cross the shockwave, but no heat is being added or subtracted from the energy in the flow. 
\end{solutionorbox}

\newpage 
\begin{question}
For the purposes of the course, we highly recommend using the tables in NACA 1135 as a tool for
performing calculations (quickly) and learning about the trends dictated by the underlying equations.
It is important to realize, however, that the tables only provide you so many significant figures. Hence,
whenever you work on an engineering problem that involves compressible flows later in your career,
please use the equations instead of the tables, because the equations provide the exact answer up to
the assumptions made to derive them.\\
a) Report the equations in NACA 1135 used to calculate the quantities in the 2nd, 3rd and 4th
columns, counting from the left;\\
b) Besides the flow assumptions (steady, adiabatic, reversible, etc.), what additional assumption
is implemented in the NACA 1135 tables? Accordingly, describe an example for which the
tables:\\
\indent i. are specialized; and\\
\indent ii. cannot be used.\\
c) Find the Mach number for which each of the entries in the 2nd through 4th column is closest to
the value 0.003, i.e. we are looking for 3 Mach numbers here, one for each column. 
\end{question}
\begin{solutionorbox}[\stretch{1}]
\end{solutionorbox}
\newpage 
\begin{question}
Find all eigenvalues and eigenfunctions for the equation: 
$\varphi'' + \lambda \varphi = 0$ with $\varphi(0) = 0$ and $\varphi'(\pi) = 0.$
\end{question}
\begin{solutionorbox}[\stretch{1}]
\end{solutionorbox}


\newpage 
\begin{question}
Solve 
\begin{align*}
\begin{cases}
\ptt u = \pxx u\\ 
u(0, t) = u(\pi, t) = 0\\ 
u(x,0) = f(x), \pt u(x,0) = 0,
\end{cases}
\end{align*}
where $f(x)$ is the ``hat function'': it's $0$ and $0$ and $\pi$ and $1$ and $\pi/2$ and 
linear in between. 
\end{question}
\begin{solutionorbox}[\stretch{1}]
\end{solutionorbox}


\newpage 
\begin{question}
This is another perspective on the SOV method. Consider the problem 
\begin{align*}
\begin{cases}
\pt u = \pxx u\\
u(0,t) = u(\pi, t) = 0\\ 
u(x,0) = g(x)
\end{cases}.
\end{align*}
For each fixed $x$, assume that the solution can be written as $\sum_{k=1}^{\infty}B_k\sin(kx)$. 
Note that the $B_k$ depend on $t$ so a better way to write it is $\sum_{k=1}^{\infty}
B_k(t) \sin(kx)$. Starting from this point, find the SOV solution. 
(Note: I had a typo that said $B_k(x)$ before; it should be 
$B_k(t)$ SORRY!!)
\end{question}
\begin{solutionorbox}[\stretch{1}]
Plugging the function into the PDE we get: 
\begin{align*}
\sum_{k=1}^{\infty}B_k'(t)\sin kx
&= \sum_{k=1}^{\infty}B_k(t)\frac{d^2}{dx^2}\sin kx
\\&= \sum_{k=1}^{\infty}B_k(t)(-k^2 \sin kx)
\\&= -\sum_{k=1}^{\infty}B_k(t)k^2 \sin kx.
\end{align*}
Subtracting the left from the right: 
\begin{align*}
0 = \sum_{k=1}^{\infty}(B_k'(t) + k^2 B_k(t))\sin kx.
\end{align*}
Since the $\sin kx$ functions are linearly independent, this implies that
for all $k$, $B_k'(t) =- k^2 B_k(t)$. So we get an ODE for $B_k$ and the 
solution is $B_k (t) = B_k(0)e^{-k^2 t}$. To find the values of the 
constants note that: 
\begin{align*}
g(x) 
= u(x,0)
= \sum_{k=1}^{\infty}B_k(0) \sin kx.
\end{align*}
That is, as before, $B_k(0) = \frac{2}{\pi}\int_{0}^{\pi}g(y)\sin ky dy$.
\end{solutionorbox}

\end{questions}
\end{document}
