\documentclass[12pt]{exam}
\usepackage[utf8]{inputenc}

\usepackage{amsmath,amstext,amsthm,amssymb,amsxtra}
\usepackage[top=1.5in, bottom=1.5in, left=1.25in, right=1.25in]	{geometry}
%\usepackage[normalem]{ulem}
\usepackage{txfonts} % pxfonts txfonts 
\usepackage[T1]{fontenc}
\usepackage{lmodern}
\renewcommand*\familydefault{\sfdefault}
 \usepackage{euler}   % better than the option below
\usepackage{pdfsync}
\usepackage{multicol}
\newcommand{\ci}[1]{_{ {}_{\scriptstyle #1}}}
\usepackage{graphicx}
\graphicspath{ {images/} }


\newcommand{\norm}[1]{\ensuremath{\left\|#1\right\|}}
\newcommand{\abs}[1]{\ensuremath{\left\vert#1\right\vert}}
\newcommand{\ip}[2]{\ensuremath{\left\langle#1,#2\right\rangle}}
\newcommand{\p}{\ensuremath{\partial}}
\newcommand{\pr}{\mathcal{P}}

\newcommand{\pbar}{\ensuremath{\bar{\partial}}}
\newcommand{\db}{\overline\partial}
\newcommand{\D}{\mathbb{D}}
\newcommand{\B}{\mathbb{B}}
\newcommand{\Sp}{\mathbb{S}}
\newcommand{\T}{\mathbb{T}}
\newcommand{\R}{\mathbb{R}}
\newcommand{\Z}{\mathbb{Z}}
\newcommand{\C}{\mathbb{C}}
\newcommand{\N}{\mathbb{N}}
\newcommand{\Q}{\mathbb{Q}}
\newcommand{\mQ}{\mathcal{Q}}
\newcommand{\mS}{\mathcal{S}}
\newcommand{\scrH}{\mathcal{H}}
\newcommand{\scrL}{\mathcal{L}}
\newcommand{\td}{\widetilde\Delta}
\newcommand{\pw}{\text{PW}}
\newcommand{\esup}{\text{ess.sup}}
\newcommand{\Tn}{\mathcal{T}_n}
\newcommand{\Bn}{\mathbb{B}_n}
\newcommand{\rt}{\mathcal{O}}
\newcommand{\avg}[1]{\langle #1 \rangle}
\newcommand{\one}{\mathbbm{1}}
\newcommand{\eps}{\varepsilon}
\newcommand{\grad}{\nabla}

\newcommand{\La}{\langle }
\newcommand{\Ra}{\rangle }
\newcommand{\rk}{\operatorname{rk}}
\newcommand{\card}{\operatorname{card}}
\newcommand{\ran}{\operatorname{Ran}}
\newcommand{\osc}{\operatorname{OSC}}
\newcommand{\im}{\operatorname{Im}}
\newcommand{\re}{\operatorname{Re}}
\newcommand{\tr}{\operatorname{tr}}
\newcommand{\vf}{\varphi}
\newcommand{\f}[2]{\ensuremath{\frac{#1}{#2}}}

\newcommand{\kzp}{k_z^{(p,\alpha)}}
\newcommand{\klp}{k_{\lambda_i}^{(p,\alpha)}}
\newcommand{\TTp}{\mathcal{T}_p}
\newcommand{\m}[1]{\mathcal{#1}}
\newcommand{\md}{\mathcal{D}}
\newcommand{\qan}{\abs{Q}^{\alpha/n}}
\newcommand{\sbump}[2]{[[ #1,#2 ]]}
\newcommand{\mbump}[2]{\lceil #1,#2 \rceil}
\newcommand{\cbump}[2]{\lfloor #1,#2 \rfloor}

\newcommand{\hn}{{8 }}
\newcommand{\dd}{{July 19}}
\newcommand{\class}{Math 412 Summer 23}
\newcommand{\term}{Spring 2022}
\newcommand{\examnum}{Homework \hn: Due \dd}
\newcommand{\examdate}{}
\newcommand{\timelimit}{75 Minutes}
\newcommand{\vc}[3]{\langle #1,#2,#3\rangle}
\newcommand*{\vv}[1]{\vec{\mkern0mu#1}}
\newcommand{\bv}[1]{\boldsymbol{#1}}
\newcommand{\hide}[1]{}
\newcommand{\uvec}[1]{\boldsymbol{\hat{\textbf{#1}}}}
\newcommand{\vex}[1]{\boldsymbol{{\textbf{#1}}}}
\newcommand{\px}{\partial_x}
\newcommand{\py}{\partial_y}
\newcommand{\pt}{\partial_t}
\newcommand{\pxx}{\partial_{xx}}
\newcommand{\pyy}{\partial_{yy}}
\newcommand{\ptt}{\partial_{tt}}


\pagestyle{head}
\firstpageheader{}{}{}
\runningheader{\class}{ Page \thepage\ of \numpages}{\examnum}
\runningheadrule

\makeatletter
\renewcommand*\env@matrix[1][*\c@MaxMatrixCols c]{%
  \hskip -\arraycolsep
  \let\@ifnextchar\new@ifnextchar
  \array{#1}}
\makeatother

%\printanswers
\begin{document}

\noindent
\begin{tabular*}{\textwidth}{l @{\extracolsep{\fill}} r @{\extracolsep{6pt}} l}
\textbf{\class} & \textbf{Name:} & \makebox[2in]{\hrulefill}\\
\end{tabular*}\\
\rule[2ex]{\textwidth}{2pt}
%

 

This is homework number \hn and it is \textbf{due on \dd}. You \textit{must} use this template for your
homework. You can either print it out and write on it and upload that, or you can use a tablet
if you have one. Alternatively, I am providing you with a link the the \LaTeX template. If you 
type your homework then (1) you'll become accustomed to using \LaTeX which is probably good 
and (2) you'll get a bonus of 2 points (each HW assignment is 10 points, so this means that
your maximum is 12 if you type). Whichever you choose, you will turn it in on Gradescope. 

Also, no matter which method you choose, your work must be neat, legible, and must flow clearly. This,
of course, includes the requirement that you show you work appropriately. 
See examples in class of what I am looking for. In particular, there should be nothing that is 
scratched out (either use Whiteout or something similar or just re-write the whole thing). The
work should more or less progress from left to right, top to bottom. In short, imagine that this
is a history class or something and you're turning in your \textit{final draft}. Part of what you 
are being graded on is your ability to communicate well which, at a minimum, means the reader can 
actually read what you have written.

Three of the problems will be graded
for accuracy, the others will be graded for ``completion''. Here, ``completion'' means: is it 
clear the student made an honest attempt at the problem and wrote the solution/attempt up in a
neat way?


\newpage 
\begin{questions}
\begin{question}
Solve 
\begin{align*}
\begin{cases}
\pt u = \pxx u\\ 
u(0, t) = u(\pi, t) = 0\\ 
u(x,0) = 2\sin 3x
\end{cases}
\end{align*}
\end{question}
\begin{solutionorbox}[\stretch{1}]
\end{solutionorbox}
\newpage 
\begin{solutionorbox}[\stretch{1}]
\end{solutionorbox}


\newpage 
\begin{question}
Find all eigenvalues and eigenfunctions for the equation: 
$\varphi'' + \lambda \varphi = 0$ with $\varphi(0) = 0$ and $\varphi'(\pi) = 0.$
\end{question}
\begin{solutionorbox}[\stretch{1}]
\end{solutionorbox}
\newpage 
\begin{solutionorbox}[\stretch{1}]
\end{solutionorbox}


\newpage 
\begin{question}
Solve 
\begin{align*}
\begin{cases}
\ptt u = \pxx u\\ 
u(0, t) = u(\pi, t) = 0\\ 
u(x,0) = f(x), \pt u(x,0) = 0,
\end{cases}
\end{align*}
where $f(x)$ is the ``hat function'': it's $0$ and $0$ and $\pi$ and $1$ and $\pi/2$ and 
linear in between. 
\end{question}
\begin{solutionorbox}[\stretch{1}]
\end{solutionorbox}
\newpage 
\begin{solutionorbox}[\stretch{1}]
\end{solutionorbox}



\newpage 
\begin{question}
This is another perspective on the SOV method. Consider the problem 
\begin{align*}
\begin{cases}
\pt u = \pxx u\\
u(0,t) = u(\pi, t) = 0\\ 
u(x,0) = g(x)
\end{cases}.
\end{align*}
For each fixed $x$, assume that the solution can be written as $\sum_{k=1}^{\infty}B_k\sin(kx)$. 
Note that the $B_k$ depend on $t$ so a better way to write it is $\sum_{k=1}^{\infty}
B_k(t) \sin(kx)$. Starting from this point, find the SOV solution. 
(Note: I had a typo that said $B_k(x)$ before; it should be 
$B_k(t)$ SORRY!!)
\end{question}
\begin{solutionorbox}[\stretch{1}]
Plugging the function into the PDE we get: 
\begin{align*}
\sum_{k=1}^{\infty}B_k'(t)\sin kx
&= \sum_{k=1}^{\infty}B_k(t)\frac{d^2}{dx^2}\sin kx
\\&= \sum_{k=1}^{\infty}B_k(t)(-k^2 \sin kx)
\\&= -\sum_{k=1}^{\infty}B_k(t)k^2 \sin kx.
\end{align*}
Subtracting the left from the right: 
\begin{align*}
0 = \sum_{k=1}^{\infty}(B_k'(t) + k^2 B_k(t))\sin kx.
\end{align*}
Since the $\sin kx$ functions are linearly independent, this implies that
for all $k$, $B_k'(t) =- k^2 B_k(t)$. So we get an ODE for $B_k$ and the 
solution is $B_k (t) = B_k(0)e^{-k^2 t}$. To find the values of the 
constants note that: 
\begin{align*}
g(x) 
= u(x,0)
= \sum_{k=1}^{\infty}B_k(0) \sin kx.
\end{align*}
That is, as before, $B_k(0) = \frac{2}{\pi}\int_{0}^{\pi}g(y)\sin ky dy$.
\end{solutionorbox}
\newpage 
\begin{solutionorbox}[\stretch{1}]
\end{solutionorbox}


\end{questions}
\end{document}
